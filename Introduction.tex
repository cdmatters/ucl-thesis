\chapter{Introduction}
\label{chapterlabel1}


Paragraph on what this chapter does or will do:
\begin{enumerate}
    \item Introduce the problem
    \item Outline motivation for the problem 
    \item Outline questions and objectives of the thesis
    \item Present the structure of the Theis 
\end{enumerate}

\section{The Problem} % (fold)
\label{sec:the_problem}

\blindtext


% section the_problem (end)
\section{Motivation} % (fold)
\label{sec:motivation}


Code documentation is a tedious and important part of writing and reading industrial code. It can be varied and describe circuituous or irrelevant information about the function. In this paper we present a new dataset of argument functions and their source code and descriptions, and show that the best methods of training data involved the use of source code as well as named variables. 
% section motivation (end)



\section{Objectives and Problem Formulation} % (fold)
\label{sec:problem_formulation}

\begin{itemize}
    \item in ideal world translate from code to english (semantic parsing)
    \item this neglects the idiomatic naturalness of big code (big code and naturalness)
    \item we seek a model of translation of elements of code and other information into natural language
    \item our investigation finds a new dataset where the link between semantic meaning, naturalness and natural language is very strong. 
    \item we investigated this dataset using machine translation techniques and found blah
\end{itemize}
% section problem_formulation (end)
 

\section{Structure of the Thesis} % (fold)
\label{sec:structure_of_the_thesis}

% section structure_of_the_thesis (end)



% section related_work (end)

 % subsection subsection_name (end) 

% Some stuff about things. \cite{example-citation} Some more things. 

% Inline citation: \bibentry{example-citation}

% This is just a bare misdnimum to get started.  There is unlimited guidance on using latex, e.g. {\tt https://en.wikibooks.org/wiki/LaTeX}.   You are still responsible to check the detailed requirements of a project, including formatting instructions, see \\
% {\tt https://moodle.ucl.ac.uk/pluginfile.php/3591429/mod\_resource/content/7/UGProjects2017.pdf}.
% Leave at least a line of white space when you want to start a new paragraph.

% Mathematical expressions are placed inline between dollar signs, e.g. $\sqrt 2, \sum_{i=0}^nf(i)$, or in display mode
% \[ e^{i\pi}=-1\] and another way, this time with labels,
% \begin{align}
% \label{line1} A=B\wedge B=C&\rightarrow A=C\\
% &\rightarrow C=A\\
% \intertext{note that}
% n!&=\prod_{1\leq i\leq n}i \\
% \int_{x=1}^y \frac 1 x \mathrm{d}x&=\log y
% \end{align}
% We can refer to labels like this \eqref{line1}. 

