\chapter{Introductory Material}
\label{chapterlabel1}

\section{Motivation} % (fold)
\label{sec:motivation}


Code documentation is a tedious and important part of writing and reading industrial code. It can be varied and describe circuituous or irrelevant information about the function. In this paper we present a new dataset of argument functions and their source code and descriptions, and show that the best methods of training data involved the use of source code as well as named variables. 
% section motivation (end)

\section{Problem Formulation} % (fold)
\label{sec:problem_formulation}

\begin{itemize}
    \item in ideal world translate from code to english (semantic parsing)
    \item this neglects the idiomatic naturalness of big code (big code and naturalness)
    \item we seek a model of translation of elements of code and other information into natural language
    \item our investigation finds a new dataset where the link between semantic meaning, naturalness and natural language is very strong. 
    \item we investigated this dataset using machine translation techniques and found blah
\end{itemize}
% section problem_formulation (end)
 
\section{Related Work} % (fold)

This could be a very big section. 

\subsubsection{Naturalness to English} % (fold)
\label{ssub:naturalness_to_english}

Here we talk a bit about the big code and naturalness papers. Allamanis etc al.

Papers such as: extreme summarization of source code, graph neural network, etc..

We reference other datasets that are available, and analyse the problems with them


\subsubsection{Code to English} % (fold)
\label{ssub:code_to_english}

Here we talk a bit about structured language to english. Semantic parsing. We talk about the datasets and the very limited fields. (Pointer networks)
% subsubsection naturalness_to_english (end)

We talk about maybe some english to code methods:
* SQL
* program synthesis

\subsection{Other Investigations with Code} % (fold)
\label{sub:other_investigations_with_code}

Here we refer to code to vec.
And maybe some more stuff
% subsection other_investigations_with_code (end)

\label{sec:related_work}

% section related_work (end)

 % subsection subsection_name (end) 

% Some stuff about things. \cite{example-citation} Some more things. 

% Inline citation: \bibentry{example-citation}

% This is just a bare misdnimum to get started.  There is unlimited guidance on using latex, e.g. {\tt https://en.wikibooks.org/wiki/LaTeX}.   You are still responsible to check the detailed requirements of a project, including formatting instructions, see \\
% {\tt https://moodle.ucl.ac.uk/pluginfile.php/3591429/mod\_resource/content/7/UGProjects2017.pdf}.
% Leave at least a line of white space when you want to start a new paragraph.

% Mathematical expressions are placed inline between dollar signs, e.g. $\sqrt 2, \sum_{i=0}^nf(i)$, or in display mode
% \[ e^{i\pi}=-1\] and another way, this time with labels,
% \begin{align}
% \label{line1} A=B\wedge B=C&\rightarrow A=C\\
% &\rightarrow C=A\\
% \intertext{note that}
% n!&=\prod_{1\leq i\leq n}i \\
% \int_{x=1}^y \frac 1 x \mathrm{d}x&=\log y
% \end{align}
% We can refer to labels like this \eqref{line1}. 

