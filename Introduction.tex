\chapter{Introduction}
\label{chapterlabel1}


% Paragraph on what this chapter does or will do:
% \begin{enumerate}
%     \item Introduce the problem
%     \item Outline motivation for the problem 
%     \item Outline questions and objectives of the thesis
%     \item Present the structure of the Thesis 
% \end{enumerate}


This chapter presents a short introduction the thesis. 
It outlines the central problem in question, motivates the research, and formalises the objectives of this investigation and report. 
Finally it presents an overview of the structure of the thesis and the content within.


% \section{The Problem} % (fold)
% \label{sec:the_problem}

% \begin{itemize}
%     \item The quality of machine translation in natural language has seen an explosive improvement in the past decade (CITE)
%     \item The proliferation of deep learning and availability of big datasets have changed the field.
%     \item However there are still other areas related to translation where 
% \end{itemize}


\section{Code and Natural Language} % (fold)
\label{sec:code_and_natural_language}

The area of machine translation is a branch of natural language processing that has seen a dramatic improvement in recent decades. \textbf{CITE}
The proliferation of `deep learning' techniqes and the availability of the large datasets have led to a series of models that have recently surpassed human ability on \textbf{CITE}. % ANOTHER SENTENCE
However, most of advances have come in the field of `natural language' - that is language as spoken, read and written by humans. 
Other modalities of communication - such as source code, as interpretted by computers - have, until recently, not been the subject of as much investigation by natual language research community.


\begin{itemize}
    \item Historically how was code treated in research commnuity
    \item Nowadays how is code treated differently, and why?
    \item Is code similar to natural language
    \item If so how can we use machine translation techniques to help understanding of source code
    \item How can we use nlp to help understanding of source code
    \item How can we use source code to help understanding of source code with nlp
\end{itemize}



\section{Motivation} % (fold)
\label{sec:motivation}

Given the prominence that computers and digital infrastructure take in modern life, it is not hard to motivate work to help humans communicate with machines, especially on their terms.
Understanding source code is already valuable skill in industry today, and one that is likely to grow in importance, as the number of people with such a skill fails match demand. 
Any work that either facilitates human understanding of source code, or helps humans to communicate to each other, via source code, is likely to find value in industry where such communication is vital.

More specifically in the realm of documentation, and translating parts of code to natural language.

\begin{itemize}
    \item The vast majority of code is undocumented, so immediate win for automation
    \item Code documentation is a tedious and important part of writing and reading industrial code, and so useful to automate.
    \item For arguments that have little type information, very hard to guess without reading source or naming so extra help. Crashes arent spotted until too late.
    \item It can be varied and describe circuituous or irrelevant information about the function.
    \item Furthermore documentation is rarely updated, so automated checking would be valuable
\end{itemize}

Finally, tangential benefits to research in the field (and therefore more datasets)

\begin{itemize}
    \item Work on type inference
    \item Understanding code patterns helps code generation thats human readable
    \item Language to code
\end{itemize}



\section{Objectives and Problem Formulation} % (fold)
\label{sec:problem_formulation}

Research hypothesise

\begin{itemize}
    \item in ideal world translate from code to english (semantic parsing)
    \item this neglects the idiomatic naturalness of big code (big code and naturalness)
    \item we seek a model of translation of elements of code and other information into natural language
    \item our investigation finds a new dataset where the link between semantic meaning, naturalness and natural language is very strong. 
    \item we investigated this dataset using machine translation techniques and found blah
\end{itemize}
% section problem_formulation (end)
 

\section{Structure of the Thesis} % (fold)
\label{sec:structure_of_the_thesis}

Over the course of this thesis we will to address all the problems listed above.
First Chapter 2 presents a brief survey of the existing literature for this field of research and its datasets.
Chapter 3 then presents a theoretical background of the approaches use in the paper. 
Chapter 4 presents an in depth examination of our new dataset, along with an analysis of its composition, while Chapter 5 presents the models which we used to investigate our central research question. 
Chapter 6 provides an in-depth report of our experiments and their results, which are analysed more substantially in Chapter 7. 
Finally we present a conclusion of the results in Chapter 8, with an elaboration on potential future work
% section structure_of_the_thesis (end)



% section related_work (end)

 % subsection subsection_name (end) 

% Some stuff about things. \cite{example-citation} Some more things. 

% Inline citation: \bibentry{example-citation}

% This is just a bare misdnimum to get started.  There is unlimited guidance on using latex, e.g. {\tt https://en.wikibooks.org/wiki/LaTeX}.   You are still responsible to check the detailed requirements of a project, including formatting instructions, see \\
% {\tt https://moodle.ucl.ac.uk/pluginfile.php/3591429/mod\_resource/content/7/UGProjects2017.pdf}.
% Leave at least a line of white space when you want to start a new paragraph.

% Mathematical expressions are placed inline between dollar signs, e.g. $\sqrt 2, \sum_{i=0}^nf(i)$, or in display mode
% \[ e^{i\pi}=-1\] and another way, this time with labels,
% \begin{align}
% \label{line1} A=B\wedge B=C&\rightarrow A=C\\
% &\rightarrow C=A\\
% \intertext{note that}
% n!&=\prod_{1\leq i\leq n}i \\
% \int_{x=1}^y \frac 1 x \mathrm{d}x&=\log y
% \end{align}
% We can refer to labels like this \eqref{line1}. 

