\chapter{Dataset Examples}
\label{example_datapoint}

% Figure \ref{lst:single_point} shows a real example of a datapoint in YAML markup. 
% Different parts of the data were used in the investigation.

\begin{table}[h!]
    \begin{center}
    \begin{tabular}{| c | c | c | c |}
        \hline
        Name &  Description     &    Field    & Type  \\
        \hline
        Argument Name & the name of the argument  & arg\_name & string \\
        Argument Description & the description of the argument & arg\_desc & string \\
        Argument Type & the annotated type for the argument & arg\_type & string or null \\
        All Arguments & all arguments used in the function & args & list of strings \\
        Other Argument Info & name, desc. and type for all other arguments & arg\_info & dict of dicts\\
        Function Name & the name of the function & name & string\\
        Signature & the function signature & sig & string\\
        Path & the full path to the file & filename & string \\
        Library & the library of the function & pkg & string\\
        Source & the source code of the function & src & string\\
        Docstring & the sphinx annotated docstring & docstring & string\\

        \hline


    \end{tabular}
    \caption {Fields collected for each single datapoint. `Field' refers to the actual key used in yaml file, and `Type' indicates the type of the data as stored in the file.}
    \label{table:metadata}
    \end{center}
\end{table}




\begin{table}[h]
    \begin{center}
    \begin{tabular}{c | c | c }       
 Arg. Name (Total) & Top 5 Descriptions & Count\\
\hline
name   (1917) &  a name for the operation (optional).                                & 1057\\
&  optional op name.                                                   & 69 \\
&  an optional variable\_scope name.                                    & 41 \\
&  a name for this operation (optional).                               & 37 \\
&  a string, the name of the layer.                                    & 31 \\

\\
\hline
x   (439) &  tensor or variable.                                                 & 32 \\
&  a tensor or variable.                                               & 17 \\
&  `bfloat16`, `half`, `float32`, `float64`, `complex64`, `com[...]   & 14 \\
&  numeric `tensor`.                                                   & 12 \\
&  array or sequence containing the data                               & 10 \\

 \\
\hline
kwargs   (303) &  additional keyword arguments which will be passed to the ap[...]   & 12 \\
& optional arguments that ``request`` takes.                          & 11 \\
& standard layer keyword arguments.                                   & 8 \\
& additional properties to be set on the :class:`~.google.clo[...]   & 7 \\
& keyword arguments to pass to base method.                           & 5 \\

\\
\hline
axis   (263)  &  axis to broadcast over                                              & 14 \\
& the dimensions to reduce. if `none` (the default), reduces [...]   & 7 \\
& the index or the name of the axis. 0 is equivalent to none [...]   & 4 \\
& a `tensor` of type `int64`.                                         & 4 \\
& the axis or axes that were summed.                                  & 4 \\

\\
\hline
dtype   (260) &  a `tf.dtype`.                                                       & 11 \\
& the data type. only floating point types are supported.             & 8 \\
& default data type for internal matrices. set to np.float32 [...]   & 7 \\
& the type of the output.                                             & 6 \\
& overrides the data type of the result.                              & 6 \\


    \end{tabular}
        \caption { A table of the most popular names with their respective most popular descriptions and counts of their occurences. These demonstrate a great variety of descriptions despite name repetition.}
    \label{table:descriptions_for_names} 
    \end{center}
\end{table}

\begin{table}[h]
    \begin{center}
    \begin{tabular}{c | c | c | c | c | c | c | c | c  }       

 Variable Name &name         & &  x            &  & kwargs   & &  axis    \\
\hline
Top 5 Libraries  &tensorflow         & 1654     & tensorflow    & 300       & tensorflow    & 62       & tensorflow    & 85 \\
&google             & 51       & matplotlib    & 43       & google        & 47       & pandas        & 61       \\
&tflearn            & 50       & scipy         & 35       & dask          & 23       & scipy         & 54       \\
&external           & 14       & tflearn       & 12       & mir\_eval     & 21       & dask          & 20       \\ 
&absl               & 13       & dask          & 9        & librosa       & 18       & librosa       & 17       \\



    \end{tabular}
        \caption { A table of the most popular names with their respective most popular packages and counts of their occurences. These demonstrate a variety of packages generating the similar names.}
    \label{table:packages_for_names} 
    \end{center}
\end{table}