\chapter{Method}
\label{the_models}

% We focused on the signature because of the important nature of this section of the code. 
% In langugages such as C++, the signature of function may be all that is presented to the user of a third-party library, when using that library. 
% Therefore being able to generate reasonable descriptions from signature items alone would already be of value. 
% This does not seem an unrealistic target, since good developer practice often involves giving insightful names to variables and functions\footnote{https://github.com/google/styleguide/blob/gh-pages/pyguide.md\#s3.16-naming}.


\section{Overview}

In our investigations we focus on two main aspects of the code to generate descriptions for our arguments. 
These are the names in the function's signature, and the \textit{variable path-contexts} (VPCs) in the function's AST. 


In our experiments, we model the names of the signature as sequences of characters. We do this to capture the fact that variable and function names can often be composite and abbreviated - e.g. \mintinline[]{python}{web_ctx}. 
In this case both parts of the name might indicate a different clue to the argument:  the  \mintinline[]{python}{web} prefix may indicate a use in the internet domain; the \mintinline[]{python}{ctx} suffix may indicate a context object. 
In designing our models, we aim to pick up these patterns and conventions on the character-level.

We also focused on generating descriptions solely from the argument's VPCs. We feel that this most robust way of drawing inferences from the code, as it only examines instructions given directly to the computer. 
Any inferences here would be invariant under transformations such as renaming of variable. 
We hoped that by examining all the VPCs present for an argument, the model would learn representation for variables that indicate common usage (such as which methods are called on it), or perhaps even type.

% We chose to model this information as bag of modified `path-contexts' \cite{alon_general_2018} relevant to each variable. This would allow us to extract only the relevant sections of code pertintent to our chosen argument from the AST. 
% We hoped our models would then be able to learn which modified `path-contexts' are most informative, picking out the most relevant sections of code.

We prepared four models to investigate these data: A Rote Learner to act as baseline for our investigations; a character-level Seq2Seq Model to investigate the signature names; an original Code2Vec Decoder model to investigate the VPCs; and a Code2Vec + Seq2Seq Model to investigate both inputs combined.

The rest of this section is dedicated to presenting each of these models, summarised in Table \ref{tab:our_models_capability}, along with the  tokenization methods to obtain our data.



\begin{table}[tb]
    \centering

    \begin{tabular}{c  c  c}
          Model & Uses Signature Data & Uses VPC Data \\ 
    \hline
    Rote Learner & \checkmark & \checkmark \\
    Seq2Seq Decoder & \checkmark & \\
    Code2Vec Decoder &    &  \checkmark \\
    Code2Vec + Seq2Seq Decoder& \checkmark & \checkmark \\
    \hline
    \end{tabular}
    \caption{Overview of Our Models}
    \label{tab:our_models_capability}
\end{table}











\section{The Models}

\subsection{Rote Learner Model} % (fold)
\label{sec:rote_learner_model}

Our Rote Learner model was designed to act as a strong benchmark in all our investigations. 
It was designed according to a simple principle: \textit{the Rote Learner generates a description from a test point by returning in full a random description from a list of best-matching training point.}
It is defined formally in Algorithm \ref{alg:rote_learner_general}. 

The benefit of this model is that the definition of `best-matching' can then be changed according to the modality of the input data. In the sections below we present a brief description of these matching algorithms, with an appendix displaying each algorithm in pseudocode. A summary of these matching algorithms is presented in Table \ref{tab:matching_summary}, 

\begin{algorithm}
    \caption{The general Rote Learner algorithm }
    \label{alg:rote_learner_general}
    \begin{algorithmic}
        \Procedure{GenerateDescription}{$t$}\Comment{Generate a description for test argument $t$ }
        \State $\mathcal{M} \gets \text{BestMatchingSet}(t, training\_points)$
        \State $x \gets \text{RandomChooseOne}(\mathcal{M})$
        \State $d \gets \text{GetDescription}(x)$
        \State \textbf{return} $d$
        \EndProcedure
    \end{algorithmic}
\end{algorithm}

\begin{enumerate}
    \item \textbf{NCharacter-Gram Overlap} \textit{(Sig. Data)} \\This matching criterion was used for function signature data, and so operates on sequences of characters. It compiles a list of the training points which have the longest n character overlap with the test data point (or the longest common substring). If the same point has two of such overlaps, it is included twice, and so forth. This pseudo code is presented in Algorithm \ref{alg:ngram_overlap}.
    \item \textbf{Proportional Contexts} \textit{(VPC Data)}\\This criterion matches VPCs by treating each VPC as an atomic unit. For each VPC within the test point, the criterion finds all the training points with the same VPC. It combines these into one list, returns this list. Naturally, the list may have duplicates of the same training point. This is equivalent to choosing the training points proportionally to how often they have an overlapping VPC with the test point. The pseudocode is presented in Algorithm \ref{alg:context_algorithm} 
    \item \textbf{Max Contexts} \textit{(VPC Data)}\\This criterion takes the list output by the \textbf{Proportional Context} criterion, and only chooses the training points that appear most frequently in the list. This is equivalent to choosing the training points that have the greatest number of overlapping path-contexts with the test point. The pseudocode is also presented in Algorithm \ref{alg:context_algorithm} 
    \item \textbf{Proportional SubContexts} \textit{(VPC Data)}\\This criterion also match VPCs of the data, but treats each VPC as sequence of nodes. Specifically, as it creates $q = p{downarrow}x_f$ and treats the new path, $q$, as the sequence of nodes and arrows that compose it. This allows it to match subpaths. The matching criterion takes the test point, and for each VPC within it, finds the training VPC with the longest subsequences that matches it. It collects the corresponding datapoints to these paths and combines them into one list. It then returns this list. This is equivalent to choosing the training points that best match each VPC in the test point, proportionally to how often they are the best match. The pseudocode is presented in Algorithm \ref{alg:sub_context_algorithm} 
    \item\textbf{Max SubContexts} \textit{(VPC Data)}\\The matching criterion takes the list output by the Proportional SubContext, and only chooses the training points that appear most frequently in the list. This is equivalent to choosing the training points that most often are the best match for all path-contexts of the test. The pseudocode is presented in Algorithm \ref{alg:sub_context_algorithm} 
    \item\textbf{[Combinations]}\textit{(Sig. Data + VPC Data)} \\ Naturally these criteria can be combined across modalities to act upon combinations of the function signature and AST. In cases where this is done, the new list is simply the combination of the lists from the combined individual criteria. 
\end{enumerate}









\begin{table}[h!]
\makebox[\linewidth][c]{
    \centering

    \begin{tabular}{l c c p{9cm}}
    \hline
    BestMatchingSet & Use Sig. & Use AST & Description: \textit{choose list of training points...} \\
    \hline
    \hline
    NCharacter-Gram Overlap     & \checkmark & & 
         which have the longest n-character gram overlap with test \\
    % \hline
    % \hline
  
    Proportional Context  & & \checkmark & 
        which match each VPC in test, and combine\\
    Max Context  & & \checkmark &
        which match each VPC in test, combine, and take most frequent\\

    Proportional SubContext  & & \checkmark &
        which \textit{best-effort} match each VPC in test, and combine\\
    Max SubContext  & & \checkmark & 
        which \textit{best-effort} each VPC in test, combine, and take most frequent\\
    \hline

    \hline
    \end{tabular}
}

    \caption{An Summary of the BestMatchingSet algorithms we use in the experiments. For sake of simplicity, a \textit{best-effort} match here refers to the match with longest matching subsequence, when treating a VPC as one long sequence}
    \label{tab:matching_summary}
\end{table}





% These chose a description from a set of training samples by:
%     \begin{itemize}
%         \item combining samples with largest subpath overlap for each path in test, and chosing from that list. (Sub Path Proportional)
%         \item combining samples with largest subpath overlap for each path in test, and chosing from the most frequent training point(s) in that list. (Sub Path Max)
%         \item combining samples full overlap for each path in test, and chosing from that list (Full Path Proportional)
%         \item combining samples full overlap for each path in test, and chosing from the most frequent training point(s) in that list. (Full Path Max)
%     \end{itemize}








% subsection rote_learner_model (end)

\subsection{Character Level Sequence to Sequence Model} % (fold)
\label{sec:character_level_sequence_to_sequence}

The character level sequence model follows the standard formulation as found in \cite{bahdanau_neural_2014}

* What is the motivation of the seq to seq model.
* Since so much effort goes into naming of variables, shouldnt there be recognisable clues? Especially since,in many languages, function signature is all thats presented in documentation.
* \_ctx => context? conv2d in the function implies somethings?
* We therefore employ a sequence to sequence model to investigate the informativeness of function signatures



% subsection character_level_sequence_to_sequence (end)

\subsection{Code2Vec Decoder Model} % (fold)
\label{sec:code2vec_decoder_model}

* We then want to investigate purely without lexical names (overlap etc)
* We decide to present a modification on Code2Vec, that is argument specific.
* The motivtion is that this hsould be able to point out only some local parts of the model, the bits of the code that are near or important might get upweighted.


% subsection code2vec_to_sequence_model (end)

\subsection{Code2Vec + LSTM Encoder} % (fold)
\label{sub:code2vec_sequence_to_sequence}

Aim to combine these models together.Natural through 

% subsubsection code2vec_sequence_to_sequence (end)
% section combined_encoder_models (end)










\section{Tokenizations} % (fold)
\label{sec:tokenizations}

\subsection{Function Signature Data}

Tokenizations of the function signature data

\subsection{Abstract Syntax Trees Data }


\subsection{Argument Description}





% subsection evaluation_procedure (end)

% subsection tokenizing_code_features (end)

% subsubsection tokenizing_argument_descriptions (end)

% subsection tokenizing_textual_input (end)

% \subsubsection{Preparing the Data} % (fold)
% \label{ssub:Preparing the Data}

% \begin{enumerate}
%     \item All our data would come from one of the partitions of the dataset described in section \ref{sec:final_preparations}.

%     \item To tokenize variable names, function names and other arguments:
%     \begin{enumerate}
%         \item we generated a vocabulary of all possible tokens in valid python
%         \item We added separator tokens to the vocabulary, \mintinline{python}{"<SEPARATOR_1>"}
%         \item We then tokenizer as sequences of characters and adding an \mintinline{python}{"<END_OF_ARG>"} token to the end.
%     \end{enumerate}
    
%     \item To tokenize the argument descriptions:
%     \begin{enumerate}
%         \item We then generated a vocabulary for our training data by:
%         \begin{enumerate}
%             \item first generating a provisional vocabulary of tokens used more than 4 times in the training data.
%             \item if these words existed in our Glove embedding vocabulary they were added to our the final vocabulary. We used the  \mintinline{python}{glove.6B.200d.txt} file of 200d embeddings trained on 6 billion tokens for our Glove embeddings, (CITE). 
%             \item when this provisional list was exhausted we took the remaining most popular words, as defined by the Glove embedding file, to fill our final vocabulary to our designated "vocab-size"
%         \end{enumerate}
%         \item We then tokenized the descriptions by moving to lower case, removing new lines, and  tokenizing them using the nltk punkt tokenizer (\mintinline{python}{nltk.word_tokenize})
%         \item We then replaced out of vocabulary tokens with an \mintinline{python}{"<UNK>"} token, and finally bookending our descriptions with \mintinline{python}{"<START>"} and \mintinline{python}{"<END>"} tokens.
%         \item Path - seq?

%     \end{enumerate}



%     \item In cases of tokenized code: 
%      \begin{enumerate}
%          \item  In the case of tokenizing code paths, we first extracted the syntax tree from the source code, and traversed it exploring all paths between two nodes.
%          \item CONTINUE
%      \end{enumerate}

% \end{enumerate}

% Once these tokenizations were complete we were ready to train our models.

% \subsubsection{Training} % (fold)
% \label{ssub:training}


%     Our models were trained in the standard way:
%     \begin{enumerate}
%         \item The Rote Learner was trained determininstically, in this case using `ngram overlap' feature, as mentioned in\ref{sec:rote_learner_model}. It was evaluated with different random seeds 50 times on the hold out dataset.
%         \item The Seq-to-Seq model was trained using the backpropagation algorithm and adaptive momentum optimisation. The hyperparameters are presented in table X. 80 Epochs were evaluated, and the best and average scores were taken \textbf{EDIT}
%     \end{enumerate}




% \subsubsection{Evaluation} % (fold)
% \label{ssub:evaluation}

% % subsection training (end
% \begin{enumerate}



%     \item Finally when it came to evaluating our models on held-out datasets, we would ran translations on the dataset and calculated the BLEU score using the BLEU method found in \textbf{HERE}.
%     In particular BLEU scores were calculated over the whole corpus and \textbf{WHAT SMOOTHING HYPERPARAMTERS}.
%     \item It should be stressed that if a hyperparameter had to be tuned, it was always done on the validation set, with the test set untouched, until final evaluation. In the case of all neural models, the results presented are using the best hyperparameters under the validation set.

% \end{enumerate}








