\maketitle
% \makedeclaration

\begin{abstract} % 300 word limit
Code documentation is the natural language component of code, designed to help humans in code comprehension. 
It is a vital part of well-written code.
Documenting fine-grained elements of code, such as individual arguments, is a laborious manual task prone to error, inaccuracy or neglect. 
Automating this task would be valuable to the software engineering industry, while providing a stepping stone to research tasks such as type inference.
However, automatic generation of such descriptions has so far not been attempted in the literature, in part due to a lack of an appropriate dataset.

In this paper we present a novel dataset sourced from most popular open-source Python libraries, to investigate the automatic documentation of function arguments from their source code. In particular we investigate generation from  two modalities of the underlying code: the names in function signature, and the structure of the abstract syntax tree. 
We find that by analysing names as a sequence of characters, we are able to generate plausible descriptions from combinations of argument names, function names and co-argument names, by both rote learning and sequence-to-sequence models. 
We also find that by representing our abstract syntax tree as a sequence of paths, we are able generate descriptions using an original architecture, our Code2Vec Decoder. We demonstrate that this model surpasses a comparable rote learner, even under partial or full occlusion of lexical data from the syntax tree.

We demonstrate that the combination of these modalities yields an improvement on each of the individual models, suggesting that multimodal models may be a promising future direction of research, and finally we investigate the challenge of applying our trained model to code from different libraries. 






\end{abstract}



\setcounter{tocdepth}{2} 


\tableofcontents
% \listoffigures
% \listoftables

% \setcounter{page}{1}
