\chapter{Literature Survey}
\label{literature_survey}

\begin{itemize}
    \item Code has become an integral part of the underpinnings of modern life. This seems likely to continue if not accelerate.
    \item However the number of people who can read \& write code remains small as a percentage of the population. 
    \item Tools that can help either read, write or analyse code hold the potential to provide great value, for both professional software engineers, and also non-coders.
    \item With the proliferation of industry quality open source code, a great deal of research has gone into the field of trying do this translation.
\end{itemize}

\section{Big Code \& Naturalness} % (fold)

\subsection{The Naturalness Hypothesis} % (fold)
\label{ssub:the_naturalness_hypothesis}

% subsubsection the_naturalness_hypothesis (end)
\begin{itemize}
    \item How does code vary from natural language?
        \begin{itemize}
            \item formal, executable, brittle, unique sentences, neologisms, reuse, ambiguity, two channels vs one channel
        \end{itemize}
    \item What similarities are there?
        \begin{itemize}
            \item naming, objects are anchored (metaphors of OOP), idiomatic writing, patterns
        \end{itemize}
    \item How does this lead to the naturalness hypothesis
        \begin{itemize}
            \item latter patterns are ignored by the computer yet present in large datasets \textbf{CITE}. 
            therefore seem to indicate that asignificant part of code is communication for other humans, not the computer - cf literate programming Knuth
            \item \textbf{Naturalness Hypothesis} - ``Software is a form of human communications; software corpora have a similar statistical properties to natural language coprora; and theses properties can be exploited to build better software tools'' 
            \item The implication is therefore we can make use of the body of work and probabilistic models on natural language and transfer to code.
        \end{itemize}
\end{itemize}


\subsection{Related Work} % (fold)
\label{ssub:related work}

A number of different people have attempted to use probabilistic models for language, in this fields ranging in a variety of tasks:

\textbf{Generating English}
\begin{itemize}
    \item Extreme summarization of source code (allamanis)
    \item Summarization using neural attention model
    \item Preidcting Programming COmmets
\end{itemize}

\textbf{Observing Patterns in Code}
\begin{itemize}
    \item Allamanis graph paper
    \item Extracting patterns/idioms from source code (allamanis)
    \item Program properties from big code
\end{itemize}

\textbf{Representation of Code}
\begin{itemize}
    \item Code2Vec
    \item Code embeddings the other paper
\end{itemize}

\textbf{Datasets}
\begin{itemize}
    \item Edinburgh NLP
    \item IFTTT
    \item Django 
\end{itemize}


% \subsection{Useful datasets} % (fold)
% \label{ssub:existing_datasets}

% Here we talk a bit about structured language to english. Semantic parsing. We talk about the datasets and the very limited fields. (Pointer networks)
% % subsubsection naturalness_to_english (end)

% We talk about maybe some english to code methods:
% * SQL
% * program synthesis

% \subsection{Other Investigations with Code} % (fold)
% \label{sub:other_investigations_with_code}

% Here we refer to code to vec.
% And maybe some more stuff
% % subsection other_investigations_with_code (end)

% \label{sec:related_work}